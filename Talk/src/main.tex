\documentclass[ucs, notheorems, handout]{beamer}

\usetheme[numbers,totalnumbers,compress, nologo]{Statmod}
\usefonttheme[onlymath]{serif}
\setbeamertemplate{navigation symbols}{}

%\mode<handout> {
%    \usepackage{pgfpages}
%    \setbeameroption{show notes}
%    \pgfpagesuselayout{2 on 1}[a4paper, border shrink=5mm]
%    \setbeamercolor{note page}{bg=white}
%    \setbeamercolor{note title}{bg=gray!10}
%    \setbeamercolor{note date}{fg=gray!10}
%}

\usepackage[utf8x]{inputenc}
\usepackage[T2A]{fontenc}
\usepackage[english, russian]{babel}
\usepackage{amsmath}
\usepackage{amsfonts}
\usepackage{tikz}
\usepackage{ragged2e}
\usepackage{wrapfig}
\usepackage{t-angles}
\usepackage{slashbox}
\usepackage{hhline}
\usepackage{multirow}
\usepackage{graphics}
\usepackage{color}
\usepackage{mathdots}
\usepackage{graphicx}
\usepackage{mdwtab}
\include{letters_series_mathbb}

\setbeamercolor{bluetext_color}{fg=blue}
\newcommand{\bluetext}[1]{{\usebeamercolor[fg]{bluetext_color}#1}}

\newtheorem{theorem}{Теорема}
\newtheorem{statement}{Утверждение}

\title[HO-MSSA]{High-order MSSA для выделения сигнала}

\author[Хромов Н.А., Голяндина Н.Э.]{Хромов Никита Андреевич, Голяндина Нина Эдуардовна}

\institute[Санкт-Петербургский Государственный Университет]{%
    \small
    Санкт-Петербургский государственный университет\\
    Кафедра статистического моделирования\\
    \vspace{1.6cm}
}

\date{Процессы управления и устойчивость\\
2 апреля 2024, Санкт-Петербург}

\subject{Talks}

\begin{document}

    \begin{frame}[plain]
        \titlepage

    \end{frame}


    \section{Введение}\label{sec:introduction}
    \begin{frame}{Постановка задачи}
        $\tX=(x_0, x_1,\ldots, x_{N-1})$, $x_i\in \mathbb{R}$ "--- вещественный временной ряд.

        $\tX = \tT + \tP + \tR$.

        $\tT$ "--- тренд, $\tP$ "--- регулярные колебания, $\tR$ "--- шум.
        \vspace{0.3cm}

        \bluetext{Возможные задачи:}
        \begin{enumerate}
            \item Выделение сигнала из ряда: нахождение $\tS = \tT + \tP$,
            \item Отделение компонент сигнала: нахождение $\tT$ и $\tP$,
            \item Нахождение параметров сигнала в параметрической модели.
        \end{enumerate}

        \vspace{0.3cm}
        Методы, основанные на подпространстве сигнала:
        \begin{itemize}
            \item SSA (задачи 1 и 2)
            \\(Golyandina et al.\ (2001), Analysis of time series structure: SSA and related techiques)
            \item ESPRIT (задача 3)
            \\(Roy, Kailath (1989), ESPRIT-estimation of signal parameters via rotational invariance techniques)
        \end{itemize}
    \end{frame}

    \begin{frame}{Пример применения SSA и ESPRIT}
        \[
            s_n = e^{-0.01 n} \cos(2\pi n / 3) + e^{-0.02 n} \cos(2\pi n / 4), \quad n=0, 1, \ldots, 46
        \]

        \smallskip
        \center
        \includegraphics[width=\textwidth]{img/decomp}
    \end{frame}

    \begin{frame}{Пример применения SSA и ESPRIT}
        \begin{center}
            \includegraphics[width=\textwidth]{img/rec}
        \end{center}
%        Восстановленные ESPRIT параметры
        \begin{table}
            \caption{Восстановленные ESPRIT параметры}
            \begin{tabular}{|c|c|c|}
                \hline
                Номер слагаемого  & 1         & 2        \\ \hline
                Период            & $3$       & $4.06$   \\ \hline
                Степень затухания & $ -0.008$ & $-0.027$ \\ \hline
            \end{tabular}
        \end{table}
%        \begin{itemize}
%            \item \bluetext{Периоды}: $T_1 = 3$, $T_2 = 4.06$
%            \item \bluetext{Степени затухания}: $\alpha_1 = -0.008$, $\alpha_2 = -0.027$
%        \end{itemize}

    \end{frame}

    \begin{frame}{Постановка задачи}
        В работе Papy et al.\ (2005) была предложена тензорная модификация метода ESPRIT и экспериментально показано её
        преимущество для конкретной модели.

        \medskip

        \textcolor{red}{Цель:} расширение предложенного Papy алгоритма для решения задачи выделения сигнала,
        исследование свойств тензорных модификаций методов семейства SSA с точки зрения
        точности выделения сигнала.

        \smallskip

        В докладе рассмотрена тензорная модификация для выделения сигнала из многомерных временных рядов.
    \end{frame}


    \section{Модель}\label{sec:model}
    \begin{frame}{Модель одномерного сигнала}
        $\tX = (x_0, x_1, \ldots, x_{N-1}) = \tS + \tR$,\\
        $\tS$ --- сигнал, $\tR$ --- шум.

        \vspace{0.4cm}

        \[
            s_n = \sum_{j=1}^{R} a_j e^{ -\alpha_j n }
            \cos\left( 2 \pi \omega_j n + \varphi_j\right)
        \]

        \vspace{0.3cm}
        Параметры:\\
        $a_j \in \mathbb{R}\setminus\{0\}$ --- амплитуды, $\alpha_j \in \mathbb{R}$ --- степени затухания,
        $\omega_j \in [0, 1/2]$ --- частоты, $\varphi_j \in [0, 2\pi)$ --- фазы.

    \end{frame}

    \begin{frame}{Модель многомерного сигнала}
%        $\tX =
%        \begin{pmatrix}
%            \tX_1\\
%            \tX_2\\
%            \vdots\\
%            \tX_P
%        \end{pmatrix}
%        =
%        \begin{pmatrix}
%            \tS_1\\
%            \tS_2\\
%            \vdots\\
%            \tS_P
%        \end{pmatrix} +
%        \begin{pmatrix}
%            \tR_1\\
%            \tR_2\\
%            \vdots\\
%            \tR_P
%        \end{pmatrix} = \tS + \tR
%        $.
        $\tX =
        \begin{pmatrix}
            \tX_1  \\
            \tX_2  \\
            \vdots \\
            \tX_P
        \end{pmatrix}
        $, $\quad\tX_p = \tS_p + \tR_p$ --- одномерные ряды.

        \vspace{0.4cm}
        Общий случай:
        \[
            s_n^{(p)} = \sum_{j=1}^{R(p)} a_j^{(p)} e^{ -\alpha_j^{(p)} n }
            \cos\left( 2 \pi \omega_j^{(p)} n + \varphi_j^{(p)}\right)
        \]
        Рассматриваемый случай:
        \[
            s_n^{(p)} = \sum_{j=1}^{R} a_j^{(p)} e^{ -\alpha_j n }
            \cos\left( 2 \pi \omega_j n + \varphi_j^{(p)}\right)
        \]
    \end{frame}


    \section{MSSA}\label{sec:mssa}
    \begin{frame}{Описание алгоритма MSSA}
        $\tX$ --- $P$-мерный временной ряд длины $N$ с сигналом $\tS$, $L<N$~--- длина окна,
        $K = N - L + 1$.

        \vspace{0.5cm}

        \bluetext{Оператор вложения одномерного ряда:}
        \[
            \mathbb{H}_L\left( \tX_p \right) =
            \begin{pmatrix}
                x_0^{(p)}     & x_1^{(p)} & { x}_2^{(p)} & \ldots        & x_{K-1}^{(p)} \\
                x_1^{(p)}     & x_2^{(p)} & \iddots      & \ldots        & \vdots        \\
                x_2^{(p)}     & \iddots   & \iddots      & \ldots        & \vdots        \\
                \vdots        & \vdots    & \vdots       & \vdots        & x_{N-2}^{(p)} \\
                x_{L-1}^{(p)} & \ldots    & \ldots       & x_{N-2}^{(p)} & x_{N-1}^{(p)} \\
            \end{pmatrix}
        \]
    \end{frame}

    \begin{frame}{Описание алгоритма MSSA}
        \bluetext{Параметры алгоритма:} $L,\, R:\: R \leqslant L < N,\, K \geqslant L$.\\
        $R$ --- число компонент, отнесённых к сигналу.

        \vspace{0.3cm}

        \bluetext{Схема алгоритма MSSA для выделения сигнала}
        \begin{enumerate}
            \item \bluetext{Вложение} $\tX \stackrel{\textcolor{red}{L}}{\mapsto} \bfH =
            [\bbH_L(\tX_1):\bbH_L(\tX_2): \dots: \bbH_L(\tX_P)]\in \bbR^{L\times KP},$

            \vspace{0.2cm}

            \item \bluetext{Разложение} $\bfH = \sum_{i=1}^{d} \sqrt{\lambda_i}U_i V_i^\rmT,\,\, d \leqslant L$

            \vspace{0.2cm}

            \item \bluetext{Группировка} $\tilde{\bfS}= \sum_{i=1}^{\textcolor{red}{R}} \sqrt{\lambda_i}U_i V_i^\rmT,\,\,
            R \leqslant d$

            \vspace{0.2cm}

            $\tilde{\bfS} = \left[ \tilde{\bfS}_1 : \tilde{\bfS}_2 : \cdots : \tilde{\bfS}_P \right]$,
            $\quad \tilde{\bfS}_p \in \bbR^{L\times K}$

            \vspace{0.2cm}

            \item \bluetext{Восстановление} Матрицы $\tilde{\bfS}_p$ усредняются вдоль побочных диагоналей:
            $\tilde{s}_n^{(p)} = \operatorname{mean}\left\{\left(\tilde{\bfS}_p\right)_{i,j}~\middle|~i + j - 2 = n\right\}.$
        \end{enumerate}
    \end{frame}

    \begin{frame}{Ранг сигнала}
        $\tS$ имеет ранг $r$, если $\forall L:~r \leqslant \min(L, K) \,\,\, \operatorname{rank}{\bbH_L(\tS)} = r$.

        Рекомендуемый выбор параметра $R$ в алгоритме: $R=r$.

        \vspace{0.4cm}

        \bluetext{Примеры}
        \begin{itemize}
            \item $s_n = A e^{\alpha n} \cos(2\pi \omega n + \varphi),$\\
            $A \ne 0,\, \alpha \in \bbR,\, \omega \in [0, 1/2],\, \varphi \in [0, 2\pi)$
            \[
                r(\omega) =
                \begin{cases}
                    1, & \omega \in \{0, 1/2\},\\
                    2, & \omega \in (0, 1/2).
                \end{cases}
            \]
            \item
            \[
                s_n^{(p)} = \sum_{j=1}^{R} a_j^{(p)} e^{ -\alpha_j n }
                \cos\left( 2 \pi \omega_j n + \varphi_j^{(p)}\right)
            \]
            $\displaystyle r = \sum_{(\omega, \alpha) \in \Omega} r(\omega)$, $\quad \Omega$ --- все уникальные пары $(\omega_j, \alpha_j)$.
        \end{itemize}
    \end{frame}


    \section{High-Order MSSA}\label{sec:high-order-mssa}
    \begin{frame}{Построение траекторного тензора}
        \center
        \includegraphics[width=\textwidth]{img/mssa-tensor-injection}

        $L < N$, $\quad K = N - L + 1 \geqslant L$
    \end{frame}

    \begin{frame}{Разложение и группировка}
        \begin{itemize}
            \item High-Order SVD траекторного тензора $\calH$ имеет вид
            \[
                \calH = \sum_{l=1}^{L} \sum_{k=1}^{K} \sum_{p=1}^{P} c_{lkp} \mathbf{U}^{(1)}_{l}
                \circ \mathbf{U}^{(2)}_{k} \circ \mathbf{U}^{(3)}_{p}.
            \]

            \vspace{0.4cm}
            \item Этап группировки в алгоритме HO-MSSA имеет вид
            \[
                \widetilde{\calH} = \sum_{l=1}^{R_1} \sum_{k=1}^{R_2} \sum_{p=1}^{R_3} c_{lkp} \mathbf{U}^{(1)}_{l}
                \circ \mathbf{U}^{(2)}_{k} \circ \mathbf{U}^{(3)}_{p},
            \]
            $R_i\leqslant \min(L, K, P)$ --- параметры алгоритма.
        \end{itemize}
    \end{frame}


    \section{Ранги сигнала в HO-MSSA}\label{sec:tensor-ranks}
    \begin{frame}{Ранги тензора}
        \bluetext{$n$-Ранг тензора:} размерность пространства, порождённого
        векторами вдоль $n$-го измерения ($\operatorname{rank}_n(\mathcal{A})$).

        \begin{theorem}
            Пусть многомерный временной ряд
            \[
                \left( x_0^{(p)}, x_1^{(p)}, \ldots, x_{N-1}^{(p)} \right), \quad p=1, 2,\ldots, P
            \]
            имеет ранг $r$ в терминах MSSA, тогда для траекторного тензора $\mathcal{H}$, построенного по любой длине окна $L<N$
            такой, что ${\min(L, K) \geqslant r}$, выполняется $\operatorname{rank}_1(\mathcal{H})=\operatorname{rank}_2(\mathcal{H})=r$,
            а $3$-ранг этого тензора равен рангу матрицы, в строках которой записаны заданные одномерные ряды.
        \end{theorem}
    \end{frame}

    \begin{frame}{Ранги сигнала в HO-MSSA}
        \begin{itemize}
            \item $1$- и $2$-ранги траекторного тензора $\calH$ сигнала $\tS$ совпадают с рангом этого сигнала в терминах MSSA.
            \item Однако ранг третьего измерения имеет иной смысл.
        \end{itemize}

        \vspace{0.2cm}

        На этапе группировки рекомендуется брать $R_1=R_2=r$ и $R_3 = r_3$, где $r$ --- MSSA-ранг сигнала,
        $r_3$ --- ранг матрицы, составленной из $\tS_p$.

        \vspace{0.2cm}

        \bluetext{Примеры}
        \[
            s_n^{(p)} = \sum_{j=1}^{R} a_j^{(p)} e^{ -\alpha_j n }\cos\left( 2 \pi \omega_j n + \varphi_j^{(p)}\right),
            \quad n\in \overline{0:N-1},
        \]
        $p\in \overline{1:P},\, a_j^{(p)}\ne 0,\, \alpha_j \in \bbR,\, \omega_j \in(0, 1/2),\, \varphi_j^{(p)}\in [0, 2\pi)$

        \vspace{0.1cm}

        \begin{enumerate}
            \item $\omega_i \ne \omega_j,\, \varphi_j^{(p)} = \varphi_j^{(m)} \hspace{0.53cm}
            \implies r = 2R,\, r_3 = R$,
            \item $\omega_i \ne \omega_j,\, \varphi_j^{(p)} = b_j p+c_j
            \implies r = 2R,\, r_3 = 2R$.
        \end{enumerate}
    \end{frame}


    \section{Численные сравнения}\label{sec:numerical-comparisons}
    \begin{frame}{Численные сравнения}
        \begin{enumerate}
            \item \bluetext{Одинаковые фазы}
            \[
                x_n^{(p)} = c_1^{(p)} e^{-0.01 n} \cos(2\pi 0.2 n) + c_2^{(p)} e^{-0.02 n} \cos(2\pi 0.22 n) + \varepsilon_n^{(p)},
            \]
            \item \bluetext{Линейно меняющиеся фазы}
            \[
                \begin{split}
                    x_n^{(p)} &= c_1^{(p)} e^{-0.01 n} \cos(2\pi 0.2 n + p \pi / 6)\\
                    &+ c_2^{(p)} e^{-0.02 n} \cos(2\pi 0.22 n + p \pi / 9) + \varepsilon_n^{(p)},
                \end{split}
            \]
        \end{enumerate}
        В обоих случаях $n=0,1,\ldots, 24$, $p=1,2,\ldots, 12$, $c_k^{(p)}\sim \rmN(0, 1)$, $\varepsilon_n^{(p)}\sim \rmN(0, 4\cdot 10^{-4})$
        и независимы.

        \vspace{0.2cm}
        Точность сравнивалась по RMSE по 1000 реализациям шума $\varepsilon_n^{(p)}$ при фиксированных $c_k^{(p)}$,
        сравнение проводилось на одних и тех же реализациях шума при выборе оптимальных
        для каждого метода параметров $L$, $R$ и $R_3$.
    \end{frame}

    \begin{frame}{Численные сравнения}
        \begin{enumerate}
            \item Равные фазы \\
            \bluetext{MSSA}: $L = 22$, $R=4$\\
            \bluetext{HO-MSSA}: $L = 20$, $R=4$, $R_3=2$
            \item Линейно меняющиеся фазы\\
            \bluetext{MSSA}: $L = 21$, $R=4$\\
            \bluetext{HO-MSSA}: $L = 21$, $R=4$, $R_3=4$
        \end{enumerate}

        \vspace{-0.4cm}
        \begin{table}[!ht]
            \centering
            \caption{RMSE оценки многомерного сигнала}
            \begin{tabular}{|l|c|c|}
                \hline
                & MSSA      & HO-MSSA   \\ \hline
                равные фазы   & $0.0107$  & $0.0079$  \\ \hline
                линейные фазы & $0.00924$ & $0.00918$ \\ \hline
            \end{tabular}
        \end{table}

        \vspace{0.2cm}
        \textcolor{red}{Вывод}: HO-MSSA выделяет многомерный сигнал с одной частотой точнее, чем MSSA.
    \end{frame}


%    \section{Выводы}\label{sec:conclusion}
%    \begin{frame}{Выводы}
%        \begin{itemize}
%            \item Тензорный вариант HO-MSSA для выделения многомерного сигнала дал точность выше, чем обычный MSSA.
%            \item
%        \end{itemize}
%    \end{frame}
\end{document}
